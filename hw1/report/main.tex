\documentclass[12pt,a4paper,oneside]{article}
\usepackage{graphicx}
\usepackage{titlepic}
\usepackage[utf8]{inputenc}
\usepackage[left=1.1in,right=1.1in, top=1in, bottom= 1in]{geometry}
\usepackage[none]{hyphenat} % Avoids to go out of margin
\usepackage{subfiles}

% Font size of figure smaller than normal size:
\usepackage{caption}
\captionsetup[figure]{font=small}
\linespread{1.2}

% --------------------------------------------- %

\title{Homework 1}	                                    % Title
\author{Flavio Maiorana 2051396}				                % Authors separated by \\
\date{\today}								    % Date

\makeatletter
\let\thetitle\@title
\let\theauthor\@author
\let\thedate\@date
\makeatother

\begin{document}

\include{title}
\newpage

When approaching a classification problem, the steps to be followed
usually start from looking at the dataset and eventually doing some actions on
it even before "cranking up" the learning algorithm.

\section{Data visualization and preprocessing}

First of all, it could be useful to gain some insight one how the dataaset is
made.

\begin{verbatim}[Dataset 1]
    N Examples: 50000
    N Inputs: 100
    N Classes: 10
    Classes: [0 1 2 3 4 5 6 7 8 9]
     - Class 0: 5000 (10.0)
     - Class 1: 5000 (10.0)
     - Class 2: 5000 (10.0)
     - Class 3: 5000 (10.0)
     - Class 4: 5000 (10.0)
     - Class 5: 5000 (10.0)
     - Class 6: 5000 (10.0)
     - Class 7: 5000 (10.0)
     - Class 8: 5000 (10.0)
     - Class 9: 5000 (10.0)
\end{verbatim}

\begin{verbatim}[Dataset 2]
    N Examples: 50000
    N Inputs: 1000
    N Classes: 10
    Classes: [0 1 2 3 4 5 6 7 8 9]
     - Class 0: 5000 (10.0)
     - Class 1: 5000 (10.0)
     - Class 2: 5000 (10.0)
     - Class 3: 5000 (10.0)
     - Class 4: 5000 (10.0)
     - Class 5: 5000 (10.0)
     - Class 6: 5000 (10.0)
     - Class 7: 5000 (10.0)
     - Class 8: 5000 (10.0)
     - Class 9: 5000 (10.0)
\end{verbatim}

\newpage

Some comments:
\begin{itemize}
    \item Both datasets have examples pertaining 10 different Classes
    \item The number of examples is pretty high 
    \item Both datasets are balanced
\end{itemize}

After that, one could want to visualize the data in order to be able to spot
some peculiarities of the data by just literally looking at it. Although, the
first problem we confront with with this specific dataset is the number of
features.

% \tableofcontents
% \newpage

% \subfile{chapters/Introduction}
% \clearpage

% \subfile{chapters/Development}
% \clearpage

% \subfile{chapters/Conclusions}
% \clearpage

\bibliographystyle{unsrt}
\bibliography{ref}
\nocite{*}  % to include references which were not cited

\end{document}